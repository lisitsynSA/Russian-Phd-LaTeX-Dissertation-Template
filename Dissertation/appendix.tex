\chapter{Программный код основных синтетических тестов}\label{app:A}

В данном приложении приводится исходный код трёх основных синтетических тестов, которые использовались для проверки бинарного оптимизатора BOLT. Основным методом получения большого размера бинарного файла (> 10 МБ) было использование рекурсивных шаблонов. С их помощью компилятор создавал множественные копии функций с разными шаблонными константами, тем самым увеличивая размер кода до нужного значения.

\begingroup
\captiondelim{ } % разделитель идентификатора с номером от наименования
\lstinputlisting[lastline=140,language={[ISO]C++},caption={Синтетический тест без стандартной библиотеки},label={lst:external1}]{listings/start.cpp}
\endgroup

\begingroup
\captiondelim{ } % разделитель идентификатора с номером от наименования
\lstinputlisting[lastline=78,language={[x86masm]Assembler},caption={Стартовый ассемблерный код для ARM архитектуры для теста без стандартной библиотеки},label={lst:external1}]{listings/start_ARM.S}
\endgroup

\begingroup
\captiondelim{ } % разделитель идентификатора с номером от наименования
\lstinputlisting[lastline=78,language={[x86masm]Assembler},caption={Стартовый ассемблерный код для x86 архитектуры для теста без стандартной библиотеки},label={lst:external1}]{listings/start_x86.S}
\endgroup

\begingroup
\captiondelim{ } % разделитель идентификатора с номером от наименования
\lstinputlisting[lastline=160,language={[ISO]C++},caption={Синтетический тест с использованием стандартной библиотеки},label={lst:external1}]{listings/1k.cpp}
\endgroup

\begingroup
\captiondelim{ } % разделитель идентификатора с номером от наименования
\lstinputlisting[lastline=192,language={[ISO]C++},caption={Синтетический тест с использованием исключений и виртуальных функций},label={lst:external1}]{listings/except+virt.cpp}
\endgroup

\chapter{Ассемблерный код демонстрационного примера}\label{app:B}
В данном приложении приводится ассемблерный код демонстрационного примера, на котором были получены профиль на x86 и ARM архитектурах.

\begingroup
\captiondelim{ } % разделитель идентификатора с номером от наименования
\lstinputlisting[lastline=264,language={[x86masm]Assembler},caption={Ассемблерный код демонстрационного примера для ARM архитектуры},label={lst:ARMobj}]{listings/ARMobjdump.s}
\label{lst:ARM}
\endgroup

\begingroup
\captiondelim{ } % разделитель идентификатора с номером от наименования
\lstinputlisting[lastline=221,language={[x86masm]Assembler},caption={Ассемблерный код демонстрационного примера для x86 архитектуры},label={lst:x86obj}]{listings/x86objdump.s}
\label{lst:X86}
\endgroup


\chapter{Акты о внедрении}\label{app:С}

    \centerfloat{
        \includegraphics[width=1.0\linewidth]{Act_teach}
    }

    \centerfloat{
        \includegraphics[width=1.0\linewidth]{Act_company}
    }