\thispagestyle{empty}

\noindent%
\begin{tabularx}{\textwidth}{@{}lXr@{}}%
    & & \large{\textbf{\textit{На правах рукописи}}}\\
    & & \large{\textsl {УДК 004.4'416}}\\
    \IfFileExists{images/logo.png}{\includegraphics[height=2.5cm]{logo}}{\rule[0pt]{0pt}{2.5cm}}  & &
    \ifnumequal{\value{showperssign}}{0}{%
        \rule[0pt]{0pt}{1.5cm}
    }\\
\end{tabularx}

\vspace{0pt plus3fill} %число перед fill = кратность относительно некоторого расстояния fill, кусками которого заполнены пустые места
\begin{center}
\textbf {\large \thesisAuthor}
\end{center}

\vspace{0pt plus1fill} %число перед fill = кратность относительно некоторого расстояния fill, кусками которого заполнены пустые места

\begin{center}
\textbf {\Large %\MakeUppercase
\thesisTitle}

\vspace{0pt plus1fill} %число перед fill = кратность относительно некоторого расстояния fill, кусками которого заполнены пустые места
{\large \textbf{Специальность: \thesisSpecialtyNumber\ "---} \textbf{\thesisSpecialtyTitle}}

\ifdefined\thesisSpecialtyTwoNumber
{\large Специальность \thesisSpecialtyTwoNumber\ "---\par \thesisSpecialtyTwoTitle}
\fi

\vspace{0pt plus1.5fill} %число перед fill = кратность относительно некоторого расстояния fill, кусками которого заполнены пустые места
\Large{\textbf{АВТОРЕФЕРАТ}}\par
\vspace{0pt plus1fill} %число перед fill = кратность относительно некоторого расстояния fill, кусками которого заполнены пустые места
\large{\textbf{диссертации на соискание учёной степени}\par \textbf{\thesisDegree}}
\end{center}

\vspace{0pt plus4fill} %число перед fill = кратность относительно некоторого расстояния fill, кусками которого заполнены пустые места
{\centering\textbf{\thesisCity~--- \thesisYear}\par}

\newpage
% оборотная сторона обложки
\thispagestyle{empty}
Работа прошла апробацию в федеральном государственном автономном образовательном учреждении высшего образования «Московский физико-технический институт (национальный исследовательский университет)»

\vspace{0.008\paperheight plus1fill}
\noindent%
\begin{tabularx}{\textwidth}{@{}lX@{}}
    \ifdefined\supervisorTwoFio
    Научные руководители:   & \supervisorRegalia\par
                              \ifdefined\supervisorDead
                              \framebox{\textbf{\supervisorFio}}
                              \else
                              \textbf{\supervisorFio}
                              \fi
                              \par
                              \vspace{0.013\paperheight}
                              \supervisorRegalia\par
                              \ifdefined\supervisorTwoDead
                              \framebox{\textbf{\supervisorTwoFio}}
                              \else
                              \textbf{\supervisorTwoFio}
                              \fi
                              \vspace{0.013\paperheight}\\
    \else
    \textbf{Научный руководитель:}   & \textbf{\supervisorRegalia}\par
                              \ifdefined\supervisorDead
                              \framebox{\textbf{\supervisorFio}}
                              \else
                              \textbf{\supervisorFio}
                              \fi
                              \vspace{0.013\paperheight}\\
    \fi
    \vspace{0.013\paperheight} \\
    \ifdefined\leadingOrganizationTitle
    \textbf{Ведущая организация:}    &
    \ifnumequal{\value{showopplead}}{0}{\vspace{6\onelineskip plus1fill}}{%
        \textbf{\leadingOrganizationTitle}
    }%
    \fi
\end{tabularx}
\vspace{0.008\paperheight plus1fill}

Защита состоится \defenseDate~на~заседании диссертационного совета \defenseCouncilNumber, созданного на базе федерального государственного автономного образовательного учреждения высшего образования <<Московский физико-технический технический институт (национальный исследовательский университет)>> (МФТИ, Физтех)

\textbf{по адресу}: 141701, Московская область, г. Долгопрудный, Институтский
переулок, д. 9.


\vspace{0.008\paperheight plus1fill}
С диссертацией можно ознакомиться в библиотеке МФТИ, Физтех и на сайте организации https://mipt.ru.

\vspace{0.008\paperheight plus1fill}
{Автореферат разослан \synopsisDate.}

\vspace{0.008\paperheight plus1fill}
\noindent%
\begin{tabularx}{\textwidth}{@{}%
>{\raggedright\arraybackslash}b{18em}@{}
>{\centering\arraybackslash}X
r
@{}}
    \textbf{Ученый секретарь}\par
    \textbf{диссертационного совета}
    &
    &
    \textbf{\defenseSecretaryFio}
\end{tabularx}
